




Département de génie chimique et de génie biotechnologique
Université de Sherbrooke
Opérations unitaires 2: GCH215					professeur: Pierre Proulx, ing.
Devoir 1, 5  février 2018, à remettre le 15 février

Remettre un fichier compressé contenant vos deux réponses (1 fichier python et un fichier chemsep) que vous nommerez  ainsi : CIP_devoir1.zip

Question 1  (à remettre : 1 fichier python : notebook ou spyder)

Pour le système 2-propanol1 / eau2 nous avons les paramètres suivants pour l’équation de Wilson :

		
		

Tracez :

P-xy et y-x pour T=353.15 K
T-xy et y-x pour P=101.325 kPa

Calculez

P et yk, pour T = 353.15 K et x1 = 0.25
P et xk, pour T = 353.15 K et y1 = 0.6
T et yk, pour P = 101.325 Kpa et x1 = 0.85
T et xk, pour T = 353.15 K et y1 = 0.4
Paz , la pression azéotropique, et x1az =  y1az, la composition azéotropique pour T=353.15 K



Question 2: (en utilisant Chemsep, à remettre un fichier Chemsep)

Effectuez le calcul de flash pour le système n-hexane/éthanol/methylcyclopentane (MCP)/benzene à 334.15 K et 1 atm :

L’alimentation est :   n-hexane(1) :0.25, ethanol(2) :0.4, MCP(3):0.2, benzene(4):0.15

Utilisez UNIFAC.



